\documentclass{article}
\usepackage[14pt]{extsizes} 
\usepackage[utf8]{inputenc}
\usepackage[russian]{babel}
\usepackage{graphicx}
\usepackage{xcolor}
\usepackage{hyperref}
\usepackage{listings}
\usepackage{float}

\usepackage{indentfirst}
\usepackage[left=2.5cm, right=1.5cm, vmargin=2.5cm]{geometry}
\lstset {language = C++, backgroundcolor = \color{blue!5}, basicstyle = \footnotesize, commentstyle={}, texcl=true}

\definecolor{linkcolor}{HTML}{799B03}
\definecolor{urlcolor}{HTML}{799B03}
\hypersetup{pdfstartview=FitH,  linkcolor = linkcolor, urlcolor=urlcolor, colorlinks=true}

\begin{document}
	
	\begin{center}
		ГУАП\\
		КАФЕДРА № 51
	\end{center}
	\begin{center}
		\begin{tabular}{clcll}
			& & & & \\
			\multicolumn{1}{l}{ПРЕПОДАВАТЕЛЬ} & \multicolumn{1}{c}{} & & & \\ \cline{1-3}
			\multicolumn{1}{|c|}{доцент, к.т.н.} & \multicolumn{1}{l|}{} & \multicolumn{1}{c|}{Линский Е.М.} & & \\ \cline{1-3}
			\multicolumn{1}{|c|}{\begin{tabular}[c]{@{}c@{}}должность , уч. степень,\\ звание\end{tabular}} & \multicolumn{1}{c|}{подпись, дата} & \multicolumn{1}{c|}{инициалы, фамилия} &  &  \\ \cline{1-3}
		\end{tabular}
	\end{center}
	
	\vspace{5cm}
	
	\begin{center}
		ОТЧЕТ ПО КУРСОВОЙ РАБОТЕ\\
		СЕТЕВАЯ ИГРА <<МОРСКОЙ БОЙ>>\\
		\vspace{1cm}
		по курсу: ТЕХНОЛОГИИ ПРОГРАММИРОВАНИЯ\\
	\end{center}
	
	\vspace{4cm}
	\begin{center}
		\begin{tabular}{cccll}
			& \multicolumn{1}{l}{} & & &  \\
			\multicolumn{1}{l}{РАБОТУ ВЫПОЛНИЛ} & & & &  \\ \cline{1-4}
			\multicolumn{1}{|c|}{СТУДЕНТ ГР. №} & \multicolumn{1}{c|}{5511} & \multicolumn{1}{c|}{} & \multicolumn{1}{c|}{Вдовенко А.} & \\ \cline{1-4}
			\multicolumn{1}{|c|}{} & \multicolumn{1}{c|}{} & \multicolumn{1}{c|}{подпись, дата} & \multicolumn{1}{c|}{\begin{tabular}[c]{@{}c@{}}инициалы,\\ фамилия\end{tabular}} &  \\ \cline{1-4}
		\end{tabular}
	\end{center}
	\vspace{1cm}
	\begin{center}
		Санкт-Петербург 2017
	\end{center}
	\thispagestyle{empty}
	\newpage
	\tableofcontents
	\newpage
	\section {Функциональная спецификация}
		\subsection{Цель работы}
			Создание клиент-серверного приложения, с помощью 
			которого возможно сыграть в <<Морской бой>> со своими друзьями, родственниками и другими пользователями.
		\subsection{Функциональность приложения}
			\begin{itemize}
				\item {Регистрация уникального никнейма для каждого игрока}
				\item {Профиль игрока}
				\item {Создание игровой комнаты для ожидания соперника}
				\item {Присоединение к уже созданной игровой комнате}
				\item {Понятный графический интерфейс во время игры}
			\end{itemize}
		\subsection{Особенности реализации}
			Для передачи данных между клиентом и сервером будет использоваться протокол TCP. Для реализации графического интерфейса клиентской части будет использована библиотека Swing. В серверной части будет реализовано многопоточность для быстрой обработки ходов всех пользователей. Проект упаковывается в jar файлы.
	\section{Руководство пользователя}
		\subsection{Просмотр профиля игрока}
			При запуске приложения пользователь должен ввести уникальный никнейм, после он может либо создать новую игровую комнату и ожидать второго игрока для начала игры, либо присоединиться к уже существующей игровой комнате и начать бой.
			\begin{figure}[H]
				\center{\includegraphics[height = 8cm]{profile}}
				\caption{Окно профиля.\label{profile}}
			\end{figure}
		\subsection{Создание игровой комнаты}
			Для создания игровой комнаты необходимо нажать кнопку <<Start game>>, отобразится окно с полями игроков.
			\begin{figure}[H]
				\center{\includegraphics[height = 6cm]{field}}
				\caption{Начало игры.\label{field}}
			\end{figure}
		\subsection{Подключиться к игровой комнате}
			Для того, чтобы присоединиться к существующей комнате, необходимо нажать кнопку <<Join>>, появится окно, в котором будут отображены имена пользователей, ожидающие второго игрока. Как только к игровой комнате подключится второй игрок, данная игровая комната не будет отображаться в списке, доступных к подключению.
			\begin{figure}[H]
				\center{\includegraphics[height = 4cm]{lobbies}}
				\caption{Доступные игровые комнаты.\label{lobbies}}
			\end{figure}
		\subsection{Игровой процесс}
			При подключении к комнате второго игрока, запускается игровой процесс. Игроки по очереди используют свои ходы, нажимая на ячейки поля оппонента. Начинает игру тот пользователь, который создал игровую комнату. У пользователя также имеется игровое табло, где отображено количество оставшихся кораблей противника и их тип.
			\begin{figure}[H]
				\center{\includegraphics[height = 4cm]{scoreboard}}
				\caption{Игровое табло.\label{scoreboard}}
			\end{figure}
		\subsection{Отображение результатов ходов}
			После проведения хода, результат отображается на поле соперника. При промахе ячейка помечается точкой, при ранении корабля ячейка меняет свой цвет на красный. При полном потоплении корабля, происходит перекрашивание ячеек в серый цвет, принадлежащие данному кораблю.
			\begin{figure}[H]
				\center{\includegraphics[height = 6cm]{playprocess}}
				\caption{Игровой процесс.\label{playprocess}}
			\end{figure}
		\section{Архитектура программы}
			Архитектура программы соответствует шаблону Client-Server-Client, представленный на рисунке \ref{clientserver}
			\begin{figure}[H]
				\center{\includegraphics[height = 6cm]{clientserver}}
				\caption{Архитектура программы.\label{clientserver}}
			\end{figure}
			Клиенты обмениваются выстрелами, посредником передачи ходов является сервер.\\
			Проект состоит из трех пакетов:
			\begin{itemize}
				\item{model}
				\item{view}
				\item{web}
			\end{itemize}
			\subsection{Model}
				Реализация логической части программы, определяющая:
				\begin{itemize}
					\item {Взаимное положение кораблей}
					\item {Осуществление выстрела}
					\item {Параметры кораблей, ячеек и игровых полей}
				\end{itemize}
			\subsection{View}
				Реализация графической части программы, отображающая профиль игрока, поля и произведенные выстрелы игроков.
			\subsection{Web}
				Реализация взаимодействия клиентов через сервер.
		\section{Основные классы и методы}
			\subsection{Model}
				Основным классом пакета является класс Field, который хранит в себе информацию о всех ячейках пользователя и кораблях.
				\begin{itemize}
					\item {public void setShip() -- случайно расставляет корабли на поле пользователя}
					\item {public int doShot(Cell input) -- производит выстрел по заданной ячейке}
					\item {public Cell getCell(int x, int y) -- возвращает объект Cell по заданным координатам}
					\item{public ArrayList<Ship> getShips() -- возвращает список кораблей}
					\item{public int getNumLiveShips() -- возвращает число оставшихся кораблей}
					\item {public int getMaxShip() -- возвращает общее количество кораблей}
				\end{itemize}
			\subsection{View}
				Пакет содержит классы, которые отображают:
				\begin{itemize}
					\item {Главное окно (класс MainFrame)}
					\item {Окно игрового поля (класс GameView)}
					\item {Игровое поле пользователя (класс PanelField)}
					\item {Игровое поле соперника (класс PanelFieldOpponent)}
					\item {Панель отображения очков (класс ScoreBoard)}
					\item {Класс, обрабатывающий нажатия мыши(класс GameController)}
				\end{itemize}
			\subsection{Web}
				Пакет содержит основной класс Client, который посылает запросы на сервер, а также принимает ответы от сервера.
				\begin{itemize}
					\item {public Cell sendShot(int x, int y) -- посылает на сервер координаты выстрела, затем принимает информацию о результате выстрела}
					\item {public Cell getShot() -- принимает информацию от сервера о ячейке, в которую противник произвел выстрел, отправляет информацию о результате выстрела}
					\item {public void sendMessage (String message) -- отправляет от имени пользователя команду на сервер}
					\item {public void newLobby() -- отправляет запрос на сервер о создании новой игровой комнаты}
					\item {public void getLobbies() -- отправляет запрос на сервер о получении списка доступных игровых комнат}
					\item {public void connect(String nameLobby) -- отправляет запрос на сервер о присоединении пользователя к определенной игровой комнате}
				\end{itemize}
		\section{Вспомогательные инструменты}
			\subsection{JAR}
				Проект был собран в jar формат. Первый файл Server.jar, который запускает сам сервер. Второй файл SeaBattle.jar, который предназначен для запуска приложения пользователем.
				\begin{figure}[H]
					\center{\includegraphics[height = 3cm]{files}}
					\caption{Упаковка в jar.\label{files}}
				\end{figure}
				\begin{figure}[H]
					\center{\includegraphics[height = 8cm]{server}}
					\caption{Пример запуска сервера с консоли.\label{server}}
				\end{figure}
			\subsection{Тесты}
				Для тестирования модели игры была использована библиотека JUnit. На рисунке \ref{junit} представлен результат тестирования класса Cell.
				\begin{figure}[H]
					\center{\includegraphics[height = 3cm]{junit}}
					\caption{Результат тестирования.\label{junit}}
				\end{figure}
				Пример исходного текста представлен на рисунке \ref{code}
				\begin{figure}[H]
					\center{\includegraphics[height = 6cm]{code}}
					\caption{Пример кода тестирования.\label{code}}
				\end{figure}
			\subsection{Github}
				Проект был расположен на Github в открытом доступе и находится по ссылке \url{https://github.com/otravlen/SeaBattle}
		\section{Заключение}
			В результате выполнения курсовой работы было реализовано клиент-серверное приложение <<Морской бой>>. Для графического отображения была использована библиотека Swing. Передача пакетов между клиентом и сервером происходит по протоколу TCP. Для удобства проект был упакован в jar файлы.
\end{document}
